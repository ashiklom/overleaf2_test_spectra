\section{Introduction} \label{s:intro}
    The biosphere as we know it is fundamentally dependent on the interactions 
    between plants and solar radiation through photosynthesis. Consequently, we 
    can learn a lot about the condition of plants by studying these 
    interactions in detail. Over the last several decades, our capability to do 
    so has expanded dramatically. At regional and global scales, we have used
% * <alexey.shiklomanov@gmail.com> 2018-04-07T17:05:04.796Z:
% 
% > At regional and global scales
% Here is a comment.
% 
% ^ <alexey.shiklomanov@gmail.com> 2018-04-07T17:05:47.246Z.
    reflectance measurements from multispectral satellites such as AVHRR, Landsat, and MODIS for mapping and monitoring of vegetation productivity,
    distribution, and abundance at high temporal frequency (e.g.
    \cite{Loveland2000, Friedl2002, Hansen2010, Houborg2015a}).  At the
    landscape scale, we have used satellites and airborne sensors with high
    spatial (e.g.  DigitalGlobe) and/or spectral (e.g.  AVIRIS) resolution --
    often in combination with laser-based instruments like LIDAR -- to gain
    insights about processes influencing canopy structure and species
    composition (\cite{Singh2015, Asner2015, Banskota2015}).  Finally, field
    spectrometers with even higher spectral resolution and spanning the entire
    visible, near infrared, and occasionally the short-wave infrared regions of
    the spectrum have provided considerable insight about leaf physiology and
    biochemistry (\cite{Serbin2012, Serbin2014, Couture2013, Zhao2014,
    Sullivan2013}).

    An important caveat of using spectral information to study vegetation is
    that the thing being measured -- reflectance (and sometimes transmittance)
    -- is usually not the variable of interest. Rather, we are interested in
    physiologically or ecologically meaningful variables such as total biomass,
    photosynthetic efficiency, species composition, or biomass, which must be
    inferred from the spectral data. Typically, this connection is made
% * <alexey.shiklomanov@gmail.com> 2017-01-11T20:49:11.044Z:
% 
% > Typically, this connection is made
% Not sure I agree?
% 
% ^ <alexey.shiklomanov@gmail.com> 2017-01-11T20:49:29.599Z:
%
% I agree though.
%
% ^.
    empirically, either by simple regression with vegetation indices
    (\cite{Haboudane2002, Huete2002, LeMaire2004, Liu2012a, Fassnacht2015} or
    through more advanced statistical methods such as partial least squares
    regression (PLSR) (\cite{Serbin2012, Couture2013, Serbin2013a}) and wavelet
    transforms (\cite{Blackburn2008, Cheng2010, Banskota2013b}).  Both
    approaches are problematic for several reasons.  First of all, the
    empirical nature of the relationships means that they are often specific to
    sensors, sites, and/or vegetation types, as evidenced by the substantial
    variability in coefficients and choice of wavelengths across studies
    (\cite{Leprieur1994, Knyazikhin1998, Myneni2002, Huete2002, Liu2012a,
    Wessels2012, Croft2014}). Furthermore, these empirical approaches are
    unable to provide a mechanistic explanation for the relationships and are
    therefore unable to distinguish true connections between reflectance
    signatures and variables of interest from spurious correlations
    (\cite{Knyazikhin2013} but see \cite{Townsend2013}). As a result, such
    approaches have limited predictive capability, for instance for use to
    inform ecosystem models (\cite{Quaife2008}). 

    Radiative transfer models (RTMs), which provide a mechanistic link between 
    plant traits and spectral signatures, are a potentially useful alternative.   
    A variety of these models exist, varying in complexity from the level of 
    individual leaves (e.g. PROSPECT -- \cite{Jacquemoud1990, Feret2008}; 
    LIBERTY -- \cite{Dawson1998, DiVittorio2009}; LEAFMOD -- 
    \cite{Ganapol1998}) to canopies (e.g. SAIL -- \cite{Verhoef1984,
    Jacquemoud2009}; ACRM \cite{Kuusk2001}; MRTM -- \cite{Wang2013}), and often
    existing as components of terrestrial ecosystem models responsible for
    determining surface energy balance and light availability for
    photosynthesis (e.g. ED2 -- \cite{Medvigy2006}; Ent DGTEM --
    \cite{Ni-Meister2010}).  In this study, we focus on the PROSPECT model,
    which has been extensively used to not only simulate synthetic vegetation
    for developing and testing new remote sensing techniques (e.g.
    \cite{LeMaire2004, Feret2011, Hunt2012, Zarco-Tejada2013, Croft2014}), but
    also for estimating leaf and canopy parameters from spectral observations
    via model inversion (e.g.  \cite{Jacquemoud1995, Jacquemoud2009, Feret2008,
    Li2011a, Atzberger2012, Zarco-Tejada2004a}).  Unfortunately, the most
    commonly used approaches for RTM inversion (e.g. least-squares
    minimization, look-up tables) fail to quantify the uncertainty or
    covariance in the resulting parameter estimates. This is problematic
    because uncertainty estimates are a fundamental requirement for drawing
    meaningful scientific conclusions from results and for assimilating results
    from multiple sources (\cite{Cressie2009, Quaife2008}). 

    The application of Bayesian statistics to RTM inversion results allows the
    quantification of uncertainty and covariance in parameter estimates and
    aids in the solution of ill-posed problems by combining multiple sources of
    information in a consistent, statistically sound way. While full Bayesian
    approaches to parameter retrieval have been reasonably popular in
    astronomical (e.g.~ \cite{Sliwa2014, Wilkman2014}) and meteorological
    contexts (e.g., \cite{DeLannoy2014, Liu2014a, Elsaesser2015}), vegetation
    RTM inversion studies typically sacrifice the ability to obtain robust
    uncertainty and covariance estimates in favor of more computationally
    efficient maximum likelihood approximations constrained by prior
    information (e.g., \cite{Yao2008, Lauvernet2008, Zhang2012, Laurent2014,
    Mousivand2014}). In this study, we build upon this work by investigating
    the information content of spectral data in the absence of independent
    prior information, focusing instead on parameter uncertainty and
    covariance.  First, we introduce a novel, fully Bayesian approach for
    radiative transfer model inversion and validate this approach on a database
    of field spectra.  Second, we simulate data using the full spectral
    response functions of ten common remote sensing platforms and investigate
    the maximum idealized accuracy and precision attainable by a RTM parameter
    retrieval using data from these platforms.  Our hope is that this work will
    pave the way towards fully utilizing the vast archive of remote sensing and
    field spectral observations to enhance our understanding of ecosystem
    processes.